\documentclass[a4paper,11pt]{article}
\usepackage{amsmath}
\usepackage[frenchb]{babel}
\usepackage[utf8]{inputenc}
\usepackage[T1]{fontenc}
\usepackage{amssymb}
\title{Université d'été - Cahier de labo}
\author{Ibanez Thomas}
\begin{document}
\maketitle
\section{Lundi 29 août 2016}
Choix du projet \newline
Préparation de la présentation \newline
Discutions sur les technologies à utiliser

\section{Mardi 30 août 2016}
Implémentation du calcul de la distance via la formule d'Haversine
\begin{equation*}
a = sin^2(\frac{\varphi_1 - \varphi_2}{2}) + cos(\varphi_1) * cos(\varphi_2) * sin^2(\frac{\lambda_1 - \lambda_2}{2})
\end{equation*}
\begin{equation*}
d = 2r *arctan\left(\frac{\sqrt{a}}{\sqrt{1-a}}\right)
\end{equation*}
Implémentation du calcul de vitesse. \newline
Utilisation d'un template pour l'application web et amélioration du design
Ajout d'une table de statistiques à la base de données \newline
Test de robustesse avec l'insertion de 2000 points \newline
Ouverture de la redirection de ports sur sampang.internet-box.ch :
\begin{itemize}
\item 8080 = Http (80)
\item 8021 = Ftp (21)
\end{itemize}
\section{Mercredi 31 août 2016}
Finition de la base de données\newline
Finition de l'application web et tests\newline
Test OK
\section{Jeudi 1 septembre 2016}
Ajout d'un hook php (android.php) pour l'application android, ce hook renvoie les entrées de la BDD en format CSV \newline

\section{Vendredi 2 septembre 2016}
Etude de l'utilisation de l'uart et création d'un fichier de référence

\section{Lundi 5 septembre 2016}
Résolution du problème du module GPS+GPRS (absence de batterie) \newline
Création du code principal (fetch gps et envoi au serveur) \newline

\section{Mardi 6 septembre 2016}
Résolution de quelques bugs (inversion lat-long et APN) \newline
Test du code: Ok !

\section{Mercredi 7 septembre 2016}
Implémentation d'un code correcteur d'erreurs via la formule suivante
Soit P un ensemble de points
\begin{equation*}
P \subset \mathbb{R}^2
\end{equation*}
On calcule la moyenne des points de P
\begin{equation*}
a = \frac{\sum\limits_{i=0}^{|P|} p_i}{|P|}
\end{equation*}
Si la distance entre un point p et la moyenne est supérieur à un seuil T, on le retire de l'ensemble
\begin{equation*}
\forall p \in P : si ||p-a|| > T  alors  P = P\backslash\{p\}
\end{equation*}
On recalcule la moyenne du nouvel ensemble
\begin{equation*}
f = \frac{\sum\limits_{i=0}^{|P|} p_i}{|P|}
\end{equation*}
Le but de ce code est d'éviter qu'un problème venant du GPS (par exemple un faux point)

\section{Jeudi 8 septembre 2016}
Ouverture des ports 5901 pour vnc et 8173 pour socket java

\section{Vendredi 9 septembre 2016}
Ajout du temps dans la table t\_rundata afin d'améliorer la précision et correction des codes php et waspmote
\end{document}