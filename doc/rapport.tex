\documentclass[a4paper,11pt]{article}
\usepackage{amsmath}
\usepackage[frenchb]{babel}
\usepackage[utf8]{inputenc}
\usepackage{fontspec} 
\usepackage[T1]{fontenc}
\usepackage{amssymb}
\title{Université d'été - Rapport}
\author{Ibanez Thomas, Lovino Maxime, Tournier Vincent}
\begin{document}
\maketitle
\section{Introduction}
Bla bla projet bla bla bla
\section{Réalisation}
\subsection{Récupération et envoi des données GPS}
La première étape pour notre application est de récupérer la position actuelle via GPS, le module GPRS+GPS de la Waspmote s'occupe de nous donner la position en degrés [-180;180].\newline
Nous appliquons ensuite un code correcteur d'erreurs afin d'éviter qu'un problème de communication n'entraine l'insertion de fausses données dans la course.\newline
Une fois la position définie, la waspmote, si une carte SIM et une connexion sont disponible, va envoyer une requête GET sur l'application web (page run.php), en définissant des flags dans l'url qui seront interprétés par le code php.
\begin{itemize}
\item uid: l'identifiant de l'utilisateur courant, (par défaut 1)
\item start: doit être défini en cas de début de course
\item x: longitude (en degrés)
\item y: latitude (en degrés)
\item cnt: numéro du point envoyé
\item time: temps (en seconde) écoulé depuis le début de la course
\item end: envoyé en fin de course
\end{itemize}
Voici un exemple de communication entre la waspmote et l'application web
\begin{itemize}
\item /run.php?uid=1\&start
\item /run.php?uid=1\&x=<longitude>\&y=<latitude>\&time=12\&cnt=1
\item /run.php?uid=1\&x=<longitude>\&y=<latitude>\&time=18\&cnt=2
\item /run.php?uid=1\&x=<longitude>\&y=<latitude>\&time=23\&cnt=3
\item \ldots
\item /run.php?uid=1\&time=2704\&end
\end{itemize}

\subsection{Traitement des données}
L'application web s'occupe du lien avec la base de données, au départ de la course un fannion running dans la ligne de l'utilisateur est défini à l'id de la course en cours.\newline
Ensuite à chaque fois qu'un point est envoyé, données du point (x,y,cnt et time) sont ajoutés dans la table t\_rundata.\newline
Lorsque l'application reçoit le fannion end, la case running de l'utilisateur est remise à 0 et les statistiques de course sont calculées.
Le calcul de la distance au sol est fait par la formule d'Haversine.
\begin{equation*}
a = sin^2(\frac{\varphi_1 - \varphi_2}{2}) + cos(\varphi_1) * cos(\varphi_2) * sin^2(\frac{\lambda_1 - \lambda_2}{2})
\end{equation*}
\begin{equation*}
d = 2r *arctan\left(\frac{\sqrt{a}}{\sqrt{1-a}}\right)
\end{equation*}
Où \newline
$\varphi_1 \; et \; \varphi_2$ sont les longitudes des deux points\newline
$\lambda_1 \; et \; \lambda_2$ sont les latitudes des deux points\newline
$r$ est le rayon de la terre (6371 Km)\newline
\subsection{Stockage des données en cas de problème de réseau}


\end{document}